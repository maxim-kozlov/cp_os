\Conclusion
    В соответствии с заданием на курсовую работу по операционным системам
    был реализован загружаемый модуль ядра операционной системы Linux.
    В процессе разработки был реализовать метод, 
    позволяющий перехватить необходимые функции и системные вызовы, 
    и логировать необходимую информацию без перехода в режим пользователя.
    Реализуемый модуль поддерживает ядра версий 5.0 для архитектуры x86\_64.

    Исследованы особенности системных вызовов осуществляющих взаимодеиствие с файловыми системами и
    различными функциями пространства пользователя для буферизованного и небуферизованного ввода/вывода.
    
    Обнаружены файлы, которые не имеют символического имени
    и выявлена проблема с перехватом sys\_open через таблицу системных вызов sys\_call\_table, 
    что может быть связано с ассемблерной оптимизацией данного 
    обработчика системного вызова.
    
    Показано, что библиотека для работы с потоками в операционной системе Linux,
    на самом деле создаёт процессы, т.к. потоки в Linux <<дорогие>>.

\pagebreak