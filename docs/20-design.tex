\chapter{Конструкторская часть}
    В данном разделе будет рассмотрена общая архитектура приложения и 
    алгоритм перехвата функций с помошью системной таблицы и ftrace.

\section{Общая архитектура приложения}
    В состав программного обеспечения входит один загружаемый модуль ядра, 
    который следит за вызовом определенных функций, 
    с последующим логированием информации 
    об аргументах и имени вызываемой функции.

\section{Перехват функций}
    TODO
    \subsection{Перехват функций через системную таблицу}
        описание, код, алгоритм
        \begin{lstlisting}[language=C, label=lst:syscall-hooking:signature, caption=Сигнатура обработчиков системных вызовов]
typedef asmlinkage long ( *syscall_t)(const struct pt_regs *);
        \end{lstlisting}
    
    \subsection{Перехват функций через ftrace}
        описание, код, алгоритм

    
        \begin{lstlisting}[language=C, label=lst:ftrace-hooking:struct, caption=Структура перехватываемой функции]
struct ftrace_hook {
    const char *name; // имя перехватываемой функции
    void *function;   // адрес функции-обертки
    void *original;   // адрес оригинальной функции

    struct ftrace_ops ops;
};
        \end{lstlisting}

        \begin{lstlisting}[language=C, label=lst:ftrace-hooking:struct, caption=Макрос для заполнения структуры перехватываемой функции]
#define HOOK(_name, _hook, _orig)   \
{                            \
    .name = (_name),         \
    .function = (_hook),     \
    .original = (_orig),     \
}
        \end{lstlisting}

\section{Связь структур}
    Получение по файловому дескриптору имени файла.
    % current->files->fdt->fd[fd]->f_path.dentry->d_iname

\section{Вывод}
    В данном разделе была рассмотрена общая архитектура приложения, 
    алгоритмы перехвата функций и 
    способ получения имени файла по файловому дескриптору.

\pagebreak