\chapter{Конструкторская часть}
    В данном разделе будет рассмотрена общая архитектура приложения и 
    методы перехвата функций с помощью системной таблицы и ftrace.

\section{Общая архитектура приложения}
    В состав программного обеспечения входит один загружаемый модуль ядра, 
    который следит за вызовом определенных функций, 
    с последующим логированием информации 
    об аргументах и имени вызываемой функции.

\section{Перехват функций}
    \subsection{Перехват функций через системную таблицу}
        Общий алгоритм установки перехвата функции через системную таблицу 
        состоит из следующих этапов:
        \begin{enumerate}
            \item поиск адреса системной таблицы;
            \item сохранение адресов оригинальных обработчиков системных вызовов;
            \item снятие защиты от модификации таблицы;
            \item модификация таблицы;
            \item восстановление защиты от записи.
        \end{enumerate}
        
        Рассмотрим каждый из этапов.

        Адрес системной таблицы можно найти с помощью функции kallsyms\_lookup\_name.
        Данная функция позволяет найти абсолютный адрес любого экспортируемого символа ядра.
        На листинге \ref{lst:syscall-hooking:kallsyms_lookup_name} приведён
        код поиска начального адреса таблицы системных вызовов.
    \begin{lstlisting}[language=C, label=lst:syscall-hooking:kallsyms_lookup_name, caption=Поиск начального адреса таблицы системных вызовов]
/* Адрес таблицы системных вызовов */
static unsigned long * __sys_call_table;

/* Поиск начального адреса таблицы системных вызовов */
__sys_call_table = kallsyms_lookup_name("sys_call_table");
    \end{lstlisting}

        Таблица системных вызовов находиться в области памяти доступной только на чтение,
        поэтому на время изменения требуется отключить глобальную защиту страниц от записи,
        изменением флага WP (Write Protection) в регистре CR0.
        Данные функции представлены в листинге \ref{lst:syscall-hooking:memory}.
        Однако встроенная в Linux функция write\_cr0() не позволяет изменять бит WP, 
        поэтому требуется своя функция, представленная в листинге \ref{lst:syscall-hooking:cr0_write}.
    
    \begin{lstlisting}[language=C, label=lst:syscall-hooking:memory, caption=Функции включение и отключение защиты от записи страницы]
#define CR0_WP 0x00010000
static inline void protect_memory(void)
{
    unsigned long cr0 = read_cr0();
    cr0_write(cr0 | CR0_WP);
}

static inline void unprotect_memory(void)
{
    unsigned long cr0 = read_cr0();
    cr0_write(cr0 & ~CR0_WP);
}
    \end{lstlisting}
    
    \begin{lstlisting}[language=C, label=lst:syscall-hooking:cr0_write, caption=Функция изменения значения регистра cr0]
extern unsigned long __force_order;
inline void cr0_write(unsigned long cr0)
{
    // mov cr0, rax
    asm volatile("mov %0, %%cr0" : "+r"(cr0), "+m"(__force_order));
}
    \end{lstlisting}
    
    В новых версиях ядра сигнатура обработчика системного вызова описывается следующим образом (листинг \ref{lst:syscall-hooking:signature}):
    \begin{lstlisting}[language=C, label=lst:syscall-hooking:signature, caption=Сигнатура обработчиков системных вызовов]
typedef asmlinkage long ( *syscall_t)(const struct pt_regs *);
    \end{lstlisting}
    где struct pt\_regs может отличаться для разных версий ядра и процессоров.
    Одно из определений представлено в листинге \ref{lst:syscall-hooking:pt_regs} \cite{linux-pt_regs}.

    \begin{lstlisting}[language=C, label=lst:syscall-hooking:pt_regs, caption=Структура регистров]
struct pt_regs {
    /*
        * NB: 32-bit x86 CPUs are inconsistent as what happens in the
        * following cases (where %seg represents a segment register):
        *
        * - pushl %seg: some do a 16-bit write and leave the high
        *   bits alone
        * - movl %seg, [mem]: some do a 16-bit write despite the movl
        * - IDT entry: some (e.g. 486) will leave the high bits of CS
        *   and (if applicable) SS undefined.
        *
        * Fortunately, x86-32 doesn't read the high bits on POP or IRET,
        * so we can just treat all of the segment registers as 16-bit
        * values.
        */
    unsigned long bx;
    unsigned long cx;
    unsigned long dx;
    unsigned long si;
    unsigned long di;
    unsigned long bp;
    unsigned long ax;
    unsigned short ds;
    unsigned short __dsh;
    unsigned short es;
    unsigned short __esh;
    unsigned short fs;
    unsigned short __fsh;
    /* On interrupt, gs and __gsh store the vector number. */
    unsigned short gs;
    unsigned short __gsh;
    /* On interrupt, this is the error code. */
    unsigned long orig_ax;
    unsigned long ip;
    unsigned short cs;
    unsigned short __csh;
    unsigned long flags;
    unsigned long sp;
    unsigned short ss;
    unsigned short __ssh;
};
    \end{lstlisting}

    Номера системных вызовов описаны в исходном коде линукса \cite{linux-nomer-syscall}.
    Зная их и начальный адрес таблицы можно получить и запомнить 
    абсолютные адреса оригинальных системных вызовов.
    После чего изменить их на функции обёртки (листинг \ref{lst:syscall-hooking:init}).

    \begin{lstlisting}[language=C, label=lst:syscall-hooking:init, caption=Модификация таблицы системных вызовов]
/* Получение адресов оригинальных системных вызовов */
orig_open   = (syscall_t)__sys_call_table[__NR_open];
orig_close  = (syscall_t)__sys_call_table[__NR_close];
orig_read   = (syscall_t)__sys_call_table[__NR_read];
orig_write  = (syscall_t)__sys_call_table[__NR_write];

/* Для модификации системной таблицы необходимо снять со страницы защиту от записи */
unprotect_memory();

/* Замена системных функций hooks */
__sys_call_table[__NR_open]     = (unsigned long)hook_open;
__sys_call_table[__NR_close]    = (unsigned long)hook_close;
__sys_call_table[__NR_read]     = (unsigned long)hook_read;
__sys_call_table[__NR_write]    = (unsigned long)hook_write;

/* Восстановить защиту от записи */
protect_memory();
    \end{lstlisting}
    
    Восстановление системных вызовов происходит аналогично перехвату,
    только в таблицу записываются изначальные адреса обработчиков (\ref{lst:syscall-hooking:exit}).

    \begin{lstlisting}[language=C, label=lst:syscall-hooking:exit, caption=Восстановление таблицы системных вызовов]
/* Восстановление системной таблицы */
unprotect_memory();
__sys_call_table[__NR_open]     = (unsigned long)orig_open;
__sys_call_table[__NR_close]    = (unsigned long)orig_close;
__sys_call_table[__NR_read]     = (unsigned long)orig_read;
__sys_call_table[__NR_write]    = (unsigned long)orig_write;
protect_memory();
    \end{lstlisting}

    \subsection{Перехват функций через ftrace}
        В листинге \ref{lst:ftrace-hooking:struct} представлена структура struct ftrace\_hook,
        которая описывает каждую перехватываемую функцию. 
        Необходимо заполнить только первые три поля: 
        имя, адрес функции-обертки и оригинальной функции.
        Остальные поля считаются деталью реализации. 
        Для повышения компактности кода рекомендуется использовать
        макросы представленные в листинге \ref{lst:ftrace-hooking:macro}.
    
        \begin{lstlisting}[language=C, label=lst:ftrace-hooking:struct, caption=Структура перехватываемой функции]
struct ftrace_hook {
    const char *name; // имя перехватываемой функции
    void *function;   // адрес функции-обертки
    void *original;   // адрес оригинальной функции

    unsigned long address;
    struct ftrace_ops ops;
};
        \end{lstlisting}

        \begin{lstlisting}[language=C, label=lst:ftrace-hooking:macro, caption=Макрос для заполнения структуры перехватываемой функции]
#define HOOK(_name, _hook, _orig)   \
{                            \
    .name = (_name),         \
    .function = (_hook),     \
    .original = (_orig),     \
}

/* массив перехватываемых функций */
static struct ftrace_hook hooks[] = {
    HOOK(<func name>, <hook func>, <original func>)
};
        \end{lstlisting}

        Для трассировки функции ядра с помощью ftrace
        необходимо сначала найти и сохранить её адрес.
        Аналогично поиску адреса системной таблицы
        для поиска адреса функции можно использовать функцию kallsyms (листинг \ref{lst:ftrace-hooking:resolve_hook_address}).

    \begin{lstlisting}[language=C, label=lst:ftrace-hooking:resolve_hook_address, caption=Поиск адреса функции по символьному имени]
/* ftrace.h */
#define MCOUNT_INSN_SIZE	4 /* sizeof mcount call */

#define USE_FENTRY_OFFSET 0
#if !USE_FENTRY_OFFSET
#pragma GCC optimize("-fno-optimize-sibling-calls")
#endif

static int fh_resolve_hook_address(struct ftrace_hook *hook)
{
    hook->address = kallsyms_lookup_name(hook->name);

    if (!hook->address)
    {
        printk(KERN_DEBUG "unresolved symbol: %s\n", hook->name);
        return -ENOENT;
    }

#if USE_FENTRY_OFFSET
    *((unsigned long*) hook->original) = hook->address + MCOUNT_INSN_SIZE;
#else
    *((unsigned long*) hook->original) = hook->address;
#endif
    return 0;
}
    \end{lstlisting}

    Недостатком ftrace является возможность бесконечной рекурсии при перехвате функции,
    в результате чего может произойти паника системы.
    Существуют два способа избежать этого:
    \begin{enumerate}
        \item обнаружить рекурсию, посмотрев на адрес возврата функции;
        \item перепрыгнуть через вызов ftrace (+ MCOUNT\_INSN\_SIZE).
    \end{enumerate}
    Для переключения между этими методами существует флаг USE\_FENTRY\_OFFSET.
    Если установлено значение 0, используется первый вариант, в противном случае -- второй.

    Если используется первый вариант, то необходимо отключить защиту, которую предоставляет ftrace.
    Она работает на сохранение регистров возврата rip, но он будет изменён нами,
    поэтому следует реализовать собственные средства защиты.
    Все сводится к тому, что в .original поле ftrace\_hook структуры 
    устанавливается адрес памяти системного вызова, указанного в .name.
    Для корректной работы необходимо указать следующие флаги:
    \begin{enumerate}
        \item FTRACE\_OPS\_FL\_IP\_MODIFY информирует ftrace, что регистр rip может быть изменён;
        \item FTRACE\_OPS\_FL\_SAVE\_REGS передавать struct pt\_regs исходного системного вызова хуку 
            (необходим для установки FTRACE\_OPS\_FL\_IP\_MODIFY);
        \item FTRACE\_OPS\_FL\_RECURSION\_SAFE отключает встроенную защиту от рекурсий.
    \end{enumerate}

    \begin{lstlisting}[language=C, label=lst:ftrace-hooking:fh_ftrace_thunk, caption=Защита от рекурсии.]
static void notrace fh_ftrace_thunk(unsigned long ip, unsigned long parent_ip, struct ftrace_ops *ops, struct pt_regs *regs)
{
    struct ftrace_hook *hook = container_of(ops, struct ftrace_hook, ops);

#if USE_FENTRY_OFFSET
    /* рекурсия не возникнет, т.к. с помощью смещения в оригинальной функции был пропущен вызов ftrace  */
    regs->ip = (unsigned long) hook->function;
#else
    /* проверка адреса возврата функции */
    if(!within_module(parent_ip, THIS_MODULE))
        regs->ip = (unsigned long) hook->function;
#endif
}
    \end{lstlisting}

    \begin{lstlisting}[language=C, label=lst:ftrace-hooking:install_hook, caption=Установка перехвата функции]
int fh_install_hook(struct ftrace_hook *hook)
{
    int err;
    err = fh_resolve_hook_address(hook);
    if(err)
        return err;

    hook->ops.func = fh_ftrace_thunk;
    hook->ops.flags = FTRACE_OPS_FL_SAVE_REGS
            | FTRACE_OPS_FL_RECURSION_SAFE
            | FTRACE_OPS_FL_IPMODIFY;

    /* вызывать fh_ftrace_thunk только тогда когда rip == hook->address */
    err = ftrace_set_filter_ip(&hook->ops, hook->address, 0, 0);
    if(err)
    {
        printk(KERN_DEBUG "ftrace_set_filter_ip() failed: %d\n", err);
        return err;
    }

    /* регистрация перехвата */
    err = register_ftrace_function(&hook->ops);
    if(err)
    {
        printk(KERN_DEBUG "register_ftrace_function() failed: %d\n", err);
        return err;
    }

    return 0;
}
    \end{lstlisting}

    При выгрузке модуля отключение перехватов происходит в обратном порядке (Листинг \ref{lst:ftrace-hooking:remove_hook}).

    \begin{lstlisting}[language=C, label=lst:ftrace-hooking:remove_hook, caption=Отключение перехвата функции]
void fh_remove_hook(struct ftrace_hook *hook)
{
    int err;
    err = unregister_ftrace_function(&hook->ops);
    if(err)
    {
        printk(KERN_DEBUG "unregister_ftrace_function() failed: %d\n", err);
    }

    err = ftrace_set_filter_ip(&hook->ops, hook->address, 1, 0);
    if(err)
    {
        printk(KERN_DEBUG "ftrace_set_filter_ip() failed: %d\n", err);
    }
}
    \end{lstlisting}

% \section{Связь структур}
%     Получение по файловому дескриптору имени файла.
%     % current->files->fdt->fd[fd]->f_path.dentry->d_iname

\section{Вывод}
    В данном разделе была рассмотрена общая архитектура приложения и
    методы перехвата функций с помощью таблицы системных вызовов и ftrace.

\pagebreak