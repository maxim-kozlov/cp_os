\chapter{Исходный код программы}
\label{chapter:appendix:1}

\begin{lstlisting}[language=C, label=lst:makefile, caption=Makefile для сбоки загружаемого модуля ядра.]
obj-m += kernel_monitor.o
moduleko-objs := kernel_monitor.o

all:
    make -C /lib/modules/$(shell uname -r)/build M=$(PWD) modules

clean:
    make -C /lib/modules/$(shell uname -r)/build M=$(PWD) clean

test:
	sudo dmesg -C
	sudo insmod kernel_monitor.ko
	-cd test-program; ./$(program).out
	-sudo rmmod kernel_monitor.ko
	dmesg | grep $(program).out
\end{lstlisting}

\begin{lstlisting}[language=C, label=lst:syscall_hooks_h, caption=syscall\_hooks.h.]
#ifndef HOOKS_H
#define HOOKS_H
#include <linux/syscalls.h>

typedef asmlinkage long ( *syscall_t)(const struct pt_regs *);

#define CR0_WP 0x00010000

extern unsigned long __force_order;
inline void cr0_write(unsigned long cr0)
{
    // mov cr0, rax
    asm volatile("mov %0, %%cr0" : "+r"(cr0), "+m"(__force_order));
}

static inline void protect_memory(void)
{
    unsigned long cr0 = read_cr0();
    cr0_write(cr0 | CR0_WP);
}

static inline void unprotect_memory(void)
{
    unsigned long cr0 = read_cr0();
    cr0_write(cr0 & ~CR0_WP);
}

#endif // HOOKS_H
\end{lstlisting}

\begin{lstlisting}[language=C, label=lst:ftrace_helper_h, caption=ftrace\_helper.h.]
#include <linux/ftrace.h>
#include <linux/linkage.h>
#include <linux/slab.h>
#include <linux/uaccess.h>
#include <linux/version.h>

#define HOOK(_name, _hook, _orig)   \
{                   \
    .name = (_name),        \
    .function = (_hook),        \
    .original = (_orig),        \
}

#define USE_FENTRY_OFFSET 0
#if !USE_FENTRY_OFFSET
#pragma GCC optimize("-fno-optimize-sibling-calls")
#endif

struct ftrace_hook {
    const char *name;
    void *function;
    void *original;

    unsigned long address;
    struct ftrace_ops ops;
};

static int fh_resolve_hook_address(struct ftrace_hook *hook)
{
    hook->address = kallsyms_lookup_name(hook->name);

    if (!hook->address)
    {
        printk(KERN_DEBUG "unresolved symbol: %s\n", hook->name);
        return -ENOENT;
    }

#if USE_FENTRY_OFFSET
    *((unsigned long*) hook->original) = hook->address + MCOUNT_INSN_SIZE;
#else
    *((unsigned long*) hook->original) = hook->address;
#endif

    return 0;
}

static void notrace fh_ftrace_thunk(unsigned long ip, unsigned long parent_ip, struct ftrace_ops *ops, struct pt_regs *regs)
{
    struct ftrace_hook *hook = container_of(ops, struct ftrace_hook, ops);

#if USE_FENTRY_OFFSET
    regs->ip = (unsigned long) hook->function;
#else
    if(!within_module(parent_ip, THIS_MODULE))
        regs->ip = (unsigned long) hook->function;
#endif
}

int fh_install_hook(struct ftrace_hook *hook)
{
    int err;
    err = fh_resolve_hook_address(hook);
    if(err)
        return err;

    hook->ops.func = fh_ftrace_thunk;
    hook->ops.flags = FTRACE_OPS_FL_SAVE_REGS
            | FTRACE_OPS_FL_RECURSION_SAFE
            | FTRACE_OPS_FL_IPMODIFY;

    err = ftrace_set_filter_ip(&hook->ops, hook->address, 0, 0);
    if(err)
    {
        printk(KERN_DEBUG "ftrace_set_filter_ip() failed: %d\n", err);
        return err;
    }

    err = register_ftrace_function(&hook->ops);
    if(err)
    {
        printk(KERN_DEBUG "register_ftrace_function() failed: %d\n", err);
        return err;
    }

    return 0;
}

void fh_remove_hook(struct ftrace_hook *hook)
{
    int err;
    err = unregister_ftrace_function(&hook->ops);
    if(err)
    {
        printk(KERN_DEBUG "rootkit: unregister_ftrace_function() failed: %d\n", err);
    }

    err = ftrace_set_filter_ip(&hook->ops, hook->address, 1, 0);
    if(err)
    {
        printk(KERN_DEBUG "rootkit: ftrace_set_filter_ip() failed: %d\n", err);
    }
}

int fh_install_hooks(struct ftrace_hook *hooks, size_t count)
{
    int err;
    size_t i;

    for (i = 0 ; i < count ; i++)
    {
        err = fh_install_hook(&hooks[i]);
        if(err)
        {
            fh_remove_hooks(hooks, i);
            return err;
        }
    }
    return 0;
}

void fh_remove_hooks(struct ftrace_hook *hooks, size_t count)
{
    size_t i;
    for (i = 0 ; i < count ; i++)
        fh_remove_hook(&hooks[i]);
}
\end{lstlisting}

\begin{lstlisting}[language=C, label=lst:kernel_monitor_c, caption=kernel\_monitor.c.]
#include <linux/init.h>
#include <linux/module.h>
#include <linux/kernel.h>
#include <linux/syscalls.h>
#include <linux/kallsyms.h>
#include <linux/namei.h>

#include <linux/module.h>
#include <linux/kernel.h>
#include <linux/syscalls.h>
#include <linux/kallsyms.h>

#include <linux/sched.h>
#include <linux/fdtable.h>
#include <linux/path.h>
#include <linux/dcache.h>

/* open.c */
struct open_flags {
    int open_flag;
    umode_t mode;
    int acc_mode;
    int intent;
    int lookup_flags;
};

#include "syscall_hooks.h"
#include "ftrace_helper.h"

MODULE_LICENSE("GPL");
MODULE_AUTHOR("Kozlov M.A.");
MODULE_DESCRIPTION("Kernel monitoring");
MODULE_VERSION("0.1");

/* Адрес таблицы системных вызовов */
static unsigned long * __sys_call_table;

/* Префикс лога */
#define KERNEL_MONITOR "[KERNEL_MONITOR]: "

syscall_t orig_open;
asmlinkage int hook_open(const struct pt_regs *regs)
{
    printk(KERN_INFO KERNEL_MONITOR "Process %d; open\n", current->pid);
    const char __user *filename = (char *)regs->di;
    int flags = (int)regs->si;
    umode_t mode = (umode_t)regs->dx;

    char kernel_filename[NAME_MAX] = {0};

    /* копировать имя файла из пр-ва пользователя в пр-во ядра */
    long error = strncpy_from_user(kernel_filename, filename, NAME_MAX);

    int fd = orig_open(regs);
        
    if (!error && current->real_parent->pid > 3)
            printk(KERN_INFO KERNEL_MONITOR "Process %d; open: %s, flags: %x; mode: %x; fd: %d\n", current->pid, kernel_filename, flags, mode, fd);

    return fd;
}

syscall_t orig_close;
asmlinkage int hook_close(const struct pt_regs *regs)
{
    unsigned int fd = (unsigned int)regs->di;

    /* Не логировать закрытие stdin, stdout, stderr */
    if (fd > 2 && current->real_parent->pid > 3)
    {
        /* Open file information: */
        // current->files

        printk(KERN_INFO KERNEL_MONITOR "Process %d; close fd: %d; filename: %s\n", current->pid, fd, 
            current->files->fdt->fd[fd]->f_path.dentry->d_iname);
    }
    return orig_close(regs);
}

syscall_t orig_read;
asmlinkage int hook_read(const struct pt_regs *regs)
{
    unsigned int fd = (unsigned int)regs->di;
    char __user *buf = (char*)regs->si;
    size_t count = (size_t)regs->dx;

    /* Не логировать стандартный ввод/вывод, а так же системные процессы */
    if (fd > 2 && current->real_parent->pid > 3)
        printk(KERN_INFO KERNEL_MONITOR "Process %d; read fd: %d; buf: %p; count: %ld; filename: %s\n", current->pid, fd, buf, count,
            current->files->fdt->fd[fd]->f_path.dentry->d_iname);
    return orig_read(regs);
}

syscall_t orig_write;
asmlinkage int hook_write(const struct pt_regs *regs)
{
    unsigned int fd = (unsigned int)regs->di;
    const char __user *buf = (const char*)regs->si;
    size_t count = (size_t)regs->dx;

    /* Не логировать стандартный ввод/вывод, а так же системные процессы */
    if (fd > 2 && current->real_parent->pid > 3)
        printk(KERN_INFO KERNEL_MONITOR "Process %d; write fd: %d; buf: %p; count: %ld; filename: %s\n", current->pid, fd, buf, count,
            current->files->fdt->fd[fd]->f_path.dentry->d_iname);
    return orig_write(regs);
}

static asmlinkage struct file* ( *orig_do_filp_open)(int dfd, struct filename *pathname, const struct open_flags *op);
static asmlinkage struct file* hook_do_filp_open(int dfd, struct filename *pathname, const struct open_flags *op)
{
    if (current->real_parent->pid > 3)
        printk(KERN_INFO KERNEL_MONITOR "Process %d; open %s;\n", current->pid, pathname->name);

    struct file* file;
    file = orig_do_filp_open(dfd, pathname, op);
    return file;
}

static asmlinkage int ( *orig_bdev_read_page)(struct block_device *bdev, sector_t sector, struct page *page);
static asmlinkage int hook_bdev_read_page(struct block_device *bdev, sector_t sector, struct page *page)
{
    int err;
    err = orig_bdev_read_page(bdev, sector, page);
    printk(KERN_INFO KERNEL_MONITOR "Process %d bdev_read_page; dev: %d\n", current->pid, bdev->bd_dev);
    return err;
}

static asmlinkage int ( *orig_bdev_write_page)(struct block_device *bdev, sector_t sector, struct page *page, struct writeback_control *wbc);
static asmlinkage int hook_bdev_write_page(struct block_device *bdev, sector_t sector, struct page *page, struct writeback_control *wbc)
{
    int err;
    err = orig_bdev_write_page(bdev, sector, page, wbc);
    printk(KERN_INFO KERNEL_MONITOR "Process %d bdev_write_page; dev: %d\n", current->pid, bdev->bd_dev);
    return err;
}

static asmlinkage ssize_t ( *orig_random_read)(struct file *file, char __user *buf, size_t nbytes, loff_t *ppos);
static asmlinkage ssize_t hook_random_read(struct file *file, char __user *buf, size_t nbytes, loff_t *ppos)
{
    /* Вызов оригинального random_read() */
    int bytes_read;
    bytes_read = orig_random_read(file, buf, nbytes, ppos);
    printk(KERN_INFO KERNEL_MONITOR "Process %d read %d bytes from /dev/random\n", current->pid, bytes_read);
    return bytes_read;
}

static asmlinkage ssize_t ( *orig_urandom_read)(struct file *file, char __user *buf, size_t nbytes, loff_t *ppos);
static asmlinkage ssize_t hook_urandom_read(struct file *file, char __user *buf, size_t nbytes, loff_t *ppos)
{
    /* Вызов оригинального urandom_read() */
    int bytes_read;
    bytes_read = orig_urandom_read(file, buf, nbytes, ppos);
    printk(KERN_INFO KERNEL_MONITOR "Process %d read %d bytes from /dev/urandom\n", current->pid, bytes_read);    
    return bytes_read;
}

static struct ftrace_hook hooks[] = 
{
    HOOK("random_read", hook_random_read, &orig_random_read),
    HOOK("urandom_read", hook_urandom_read, &orig_urandom_read),
    HOOK("bdev_read_page", hook_bdev_read_page, &orig_bdev_read_page),
    HOOK("bdev_write_page", hook_bdev_write_page, &orig_bdev_write_page),
    HOOK("do_filp_open", hook_do_filp_open, &orig_do_filp_open)
};

/* Функция инициализации модуля */
static int __init kernel_monitor_init(void)
{
    /* Поиск начального адреса таблицы системных вызовов */
    __sys_call_table = kallsyms_lookup_name("sys_call_table");
    if (!__sys_call_table)
        return -1;
    
    int err;
    /* Установка ftrace hooks */
    err = fh_install_hooks(hooks, ARRAY_SIZE(hooks));
    if (err)
        return err;

    /* Получение адресов оригинальных системных вызовов */
    orig_mkdir  = (syscall_t)__sys_call_table[__NR_mkdir];
    
    orig_open   = (syscall_t)__sys_call_table[__NR_open];
    orig_close  = (syscall_t)__sys_call_table[__NR_close];
    orig_read   = (syscall_t)__sys_call_table[__NR_read];
    orig_write  = (syscall_t)__sys_call_table[__NR_write];

    /* Логгирование адресов системных вызовов */
    printk(KERN_INFO KERNEL_MONITOR "Loading ...");
    printk(KERN_DEBUG KERNEL_MONITOR "Found the syscall table at 0x%lx\n", __sys_call_table);
    printk(KERN_DEBUG KERNEL_MONITOR "mkdir: 0x%lx\n", orig_mkdir);
    printk(KERN_DEBUG KERNEL_MONITOR "open:  0x%lx\n", orig_open);
    printk(KERN_DEBUG KERNEL_MONITOR "close: 0x%lx\n", orig_close);
    printk(KERN_DEBUG KERNEL_MONITOR "read:  0x%lx\n", orig_read);
    printk(KERN_DEBUG KERNEL_MONITOR "write: 0x%lx\n", orig_write);

    printk(KERN_INFO KERNEL_MONITOR "Hooking syscalls\n");
    
    /* Для модификации системной таблицы необходимо снять со страницы защиту от записи */
    unprotect_memory();
    __sys_call_table[__NR_open]     = (unsigned long)hook_open;
    __sys_call_table[__NR_close]    = (unsigned long)hook_close;
    __sys_call_table[__NR_read]     = (unsigned long)hook_read;
    __sys_call_table[__NR_write]    = (unsigned long)hook_write;
    /* Восстановить защиту от записи */
    protect_memory();

    return 0;
}

static void __exit kernel_monitor_exit(void)
{
    printk(KERN_INFO KERNEL_MONITOR "Restoring syscalls\n");

    /* Удаление hooks ftrace */
    fh_remove_hooks(hooks, ARRAY_SIZE(hooks));

    /* Восстановление системной таблицы */
    unprotect_memory();
    __sys_call_table[__NR_open]     = (unsigned long)orig_open;
    __sys_call_table[__NR_close]    = (unsigned long)orig_close;
    __sys_call_table[__NR_read]     = (unsigned long)orig_read;
    __sys_call_table[__NR_write]    = (unsigned long)orig_write;
    protect_memory();
    
    printk(KERN_INFO KERNEL_MONITOR "Unloaded.\n");
}

module_init(kernel_monitor_init);
module_exit(kernel_monitor_exit); 
\end{lstlisting}

\pagebreak